\documentclass[a4paper,12pt]{article}
\usepackage{fullpage}

\usepackage[T2A]{fontenc}
\usepackage[utf8]{inputenc}
\usepackage[russian]{babel}

\usepackage{amsmath}
\usepackage{physics}
\usepackage{derivative}
\usepackage{hyperref}

\usepackage{csvsimple,longtable,booktabs}
\usepackage{multicol}
\usepackage{supertabular}
\usepackage{pdflscape}
\usepackage{listings}

\usepackage{graphicx}
\usepackage{subfig}
\graphicspath{ {./images/} }

\makeatletter
\let\orignewpage\newpage
\newcommand{\changenewpage}{%
    \renewcommand\newpage{%
        \if@firstcolumn
            \hrule width\linewidth height0pt
            \columnbreak
        \else
            \orignewpage
        \fi
    }
}
\makeatother

\title{Тестовое задание для компании ООО "РИТМ"}
\author{В.~Шаршуков}

\begin{document}

\maketitle
\pagebreak

\tableofcontents
\pagebreak

\section{Формулировка}

Написать программу численного решения задачи Коши для уравнения:
\begin{equation}\label{eq:ode}
    \begin{gathered}
        y^{(5)} + 15y^{(4)} + 90y''' + 270y'' + 405y' + 243y = 0,
            \quad x \in [0;5], \\
        y(0) = 0, \quad y'(0) = 3, \quad y''(0) = -9,
            \quad y'''(0) = -8, \quad y^{(4)}(0) = 0.
    \end{gathered}
\end{equation}

\begin{enumerate}
    \item Реализовать какую-либо численную схему
        \textbf{без использования готовых решений}.

    \item Построить график решения.

        Допустимо использовать сторонние средства построения графиков:
        \texttt{gnuplot, Excel}, etc.

    \item Обосновать достоверность полученного результата.
\end{enumerate}

\pagebreak

\section{Аналитическое решение}

Уравнение \ref{eq:ode} является линейным однородным дифференциальным уравнением
пятого порядка с постоянными коэффициентами, поэтому для нахождения общего
решения составим характеристическое уравнение:
\begin{equation*}
    \lambda^5 + 15 \lambda^4 + 90 \lambda^3 + 270 \lambda^2 + 405 \lambda
        + 243 = 0.
\end{equation*}

Замечая, что
\begin{equation*}
    15 = 5 \cdot 3, \quad 90 = 10 \cdot 3^2, \quad 270 = 10 \cdot 3^3,
        \quad 405 = 5 \cdot 3^4, \quad 243 = 3^5,
\end{equation*}
приходим к уравнению
\begin{equation*}
    (\lambda + 3)^5 = 0,
\end{equation*}
откуда следует, что характеристическое уравнение имеет ровно один корень
$\lambda = -3$ с кратностью 5, поэтому общее решение исходного дифференциального
уравнения представимо в виде
\begin{equation*}
    y = (C_0 + C_1 x + C_2 x^2 + C_3 x^3 + C_4 x^4) e^{-3x}.
\end{equation*}

Найдём теперь производные до 4 порядка включительно:
\begin{equation*}
    \begin{aligned}
        y' &= \phantom{-3} (C_1 + 2 C_2 x + 3 C_3 x^2 + 4 C_4 x^3) e^{-3x} \\
           & \phantom{= } -3
           \underbrace{
               (C_0 + C_1 x + C_2 x^2 + C_3 x^3 + C_4 x^4) e^{-3x}
           }_{\displaystyle y} \\
           &= (C_1 + 2 C_2 x + 3 C_3 x^2 + 4 C_4 x^3) e^{-3x} - 3y. \\
    \end{aligned}
\end{equation*}

\begin{equation*}
    \begin{aligned}
        y'' &= \phantom{-3} (2 C_2 + 6 C_3 x + 12 C_4 x^2) e^{-3x} \\
           & \phantom{= } -3
           \underbrace{
               (C_1 + 2 C_2 x + 3 C_3 x^2 + 4 C_4 x^3) e^{-3x}
           }_{\displaystyle y' + 3y} - 3y' \\
           &= (2 C_2 + 6 C_3 x + 12 C_4 x^2) e^{-3x} - 6y' - 9y.
    \end{aligned}
\end{equation*}

\begin{equation*}
    \begin{aligned}
        y''' &= \phantom{-3} (6 C_3 + 24 C_4 x) e^{-3x} \\
           & \phantom{= } -3
           \underbrace{
               (2 C_2 + 6 C_3 x + 12 C_4 x^2) e^{-3x}
           }_{\displaystyle y'' + 6y' + 9y} - 6y'' - 9y' \\
           &= (6 C_3 + 24 C_4 x) e^{-3x} - 9y'' - 27y' - 27y.
    \end{aligned}
\end{equation*}

\begin{equation*}
    \begin{aligned}
        y^{(4)} &= 24 C_4 e^{-3x} -3
           \underbrace{
               (6 C_3 + 24 C_4 x) e^{-3x}
           }_{\displaystyle y''' + 9y'' + 27y' + 27y} - 9y''' - 27y'' - 27y' \\
           &= 24 C_4 e^{-3x} - 12 y''' - 54 y'' - 108 y' - 81y.
    \end{aligned}
\end{equation*}

Пользуясь начальными условиями \ref{eq:ode}, определим значения констант:
\begin{equation*}
    \begin{aligned}
        &y(0) = 0, &&\implies && C_0 = 0. \\
        &y'(0) = 3, &&\implies && C_1 = 3. \\
        &y''(0) = -9, &&\implies -9 = 2 C_2 - 18, &&
            C_2 = \dfrac{9}{2}. \\
        &y'''(0) = -8, &&\implies -8 = 6 C_3 + 81 - 81, &&
            C_3 = -\dfrac{4}{3}. \\
        &y^{(4)}(0) = 0, &&\implies 0 = 24 C_4 + 96 + 486 - 324, &&
            C_4 = -\dfrac{43}{4}
    \end{aligned}
\end{equation*}

Итак, искомое решение задачи Коши:
\begin{equation}
    \begin{aligned}
        y_1 &= e^{-3x}
            (3x + \frac{9}{2} x^2 - \frac{4}{3} x^3 - \frac{43}{4} x^4) \\
            &= -\frac{1}{12} x e^{-3x} (129 x^3 + 16 x^2 - 54 x - 36).
    \end{aligned}
\end{equation}

\pagebreak

\section{Решение методом Рунге-Кутты}

Для начала, произведя замену переменных
\begin{equation*}
    y_1 = y, \quad y_2 = y' = y_1', \quad y_3 = y'' = y_2',
        \quad y_4 = y''' = y_3', \quad y_5 = y^{(4)} = y_4',
\end{equation*}
запишем дифференциальное уравнение \ref{eq:ode} в виде системы:
\begin{equation*}
    \begin{cases}
        y_1' = y_2, \\
        y_2' = y_3, \\
        y_3' = y_4, \\
        y_4' = y_5, \\
        y_5' = -243 y_1 - 405 y_2 - 270 y_3 - 90 y_4 - 15 y_5.
    \end{cases}
\end{equation*}
Для удобства введём обозначения
\begin{equation*}
    \vec{y} =
    \left(
    \begin{array}{c}
        y_1 \\
        y_2 \\
        y_3 \\
        y_4 \\
        y_5
    \end{array}
    \right),
    \quad
    A = 
    \left(
    \begin{array}{c c c c c}
        0 & 1 & 0 & 0 & 0 \\
        0 & 0 & 1 & 0 & 0 \\
        0 & 0 & 0 & 1 & 0 \\
        0 & 0 & 0 & 0 & 1 \\
        -243 & -405 & -270 & -90 & -15
    \end{array}
    \right);
\end{equation*}
тогда система дифференциальных уравнений запишется в векторной форме:
\begin{equation*}
    \dv{\vec{y}}{x} = \vec{f}(x, \vec{y}) = A \vec{y}.
\end{equation*}

В качестве численного метода решения задачи Коши возьмём метод Рунге-Кутты
четвёртого порядка с постоянным шагом $h$ и итерационной формулой
\begin{equation*}
    \vec{y}_{i+1} = \vec{y}_i + \frac{1}{6}
    \left(
        \vec{k}_1 + 2 \vec{k}_2 + 2 \vec{k}_3 + \vec{k}_4
    \right),
\end{equation*}
где
\begin{equation*}
    \begin{aligned}
        \vec{k}_1 &= h \vec{f}(x_i, \vec{y}_i) &&= h A \vec{y}_i, \\
        \vec{k}_2
            &= h \vec{f}(x_i + \frac{h}{2}, \vec{y}_i + \frac{1}{2}\vec{k}_1)
            &&= h A (\vec{y}_i + \frac{1}{2}\vec{k}_1), \\
        \vec{k}_3
            &= h \vec{f}(x_i + \frac{h}{2}, \vec{y}_i + \frac{1}{2}\vec{k}_2)
            &&= h A (\vec{y}_i + \frac{1}{2}\vec{k}_2), \\
        \vec{k}_4
            &= h \vec{f}(x_i + h, \vec{y}_i + \vec{k}_3)
            &&= h A (\vec{y}_i + \vec{k}_3).
    \end{aligned}
\end{equation*}

Начальные условия: $x_0 = 0, \quad \vec{y}_0 = (0, 3, -9, -8, 0)^T$.

\pagebreak

\section{Графики решений}

Введём обозначения:
\begin{itemize}
    \item $y_n(x)$ -- решение, полученное с помощью численного метода.
    \item $y_e(x)$ -- решение, полученное аналитически.
\end{itemize}

\begin{figure}[h]
    \centering
    \subfloat[$y_n(x)$]{\includegraphics[width=0.4\textwidth]{y_num}}
    \subfloat[$y_e(x)$]{\includegraphics[width=0.4\textwidth]{y_etalon}} \\
    \subfloat[Сравнение]{\includegraphics[width=0.4\textwidth]{plot}}
    \caption{Графики решений}
\end{figure}

\pagebreak

\include{data/points}

\end{document}

